% molde para cartas con membrete Dpto. y para sobres con ventana direcci¢n
\documentclass{letter}
\usepackage{graphicx}
\usepackage[spanish]{babel}      % castellano
\usepackage[latin1]{inputenc}    % tildes
\usepackage{url}

% para cartas en ingl‚s comentar la linea anterior e incluir la siguiente:
%\documentstyle[12pt]{a4letter}
% cabecera para hojas con membrete del Departamento y JJ
% Set left margin - The default is 1 inch, so the following 
% command sets a 0.5-inch left margin.
\setlength{\oddsidemargin}{0.0in}

% Set width of the text - What is left will be the right margin.
% In this case, right margin is 8.5in - 0.0in - 6.5in = 2.0in.
\setlength{\textwidth}{6.5in}

% Set top margin - The default is 1 inch, so the following 
% command sets a 0.5-inch top margin.
\setlength{\topmargin}{-0.75in}

% Set height of the text - What is left will be the bottom margin.
% In this case, bottom margin is 11in - 0.75in - 9.5in = 0.75in
\setlength{\textheight}{10in}

% cambiar estas definiciones para personalizar el documento
\def\yo{\sc Juan J. Merelo}
\def\email{\tt jmerelo@ugr.es}
\def\tel{+34-958-243162}

% definiciones comunes
\def\granada{18071 Granada (Spain)}
\def\ucm{Universidad de Granada}
\def\fac{ETS Ingenierías Informática y Telecomucnicaciones}
\def\dpto{Departamento de Arquitectura y Tecnolog\'ia de Computadores}

\def\cabecera{

\begin{minipage}{2.5cm}
  \centerline{  \includegraphics[width=2.5cm,height=2.5cm]{UGR-MARCA-01-color} }
\end{minipage}
\ \
\begin{minipage}{13cm}
 \rule{13cm}{0.6mm}
 \vspace*{0.3cm}
 {\small{\bf \dpto \hfill}} \\
 {\footnotesize{\bf \ucm \hfill Tel. / Phone: \tel}} \\
 {\footnotesize{\bf \granada \hfill E-mail: \email}}\\
 \rule{13cm}{0.6mm}
\end{minipage}
}

%\input cabecera
\pagestyle{empty}

\begin{document}

\cabecera 

\vspace{0.5cm}

\hspace{7cm}

\hfill Granada, \today

\vspace{0.5cm}
\ \
% **** si no se quiere recuadro en la dirección, comentar la linea siguiente
%\framebox[9cm][l]
{
\begin{minipage}[t]{9cm}
Dear Histories editor(s):
% ******* Poner aquí nombre y direcci¢n del destinatario
\end{minipage}
 }

\vspace{0.5cm}



\begin{flushleft}

I have submitted separately the manuscript ``It's a doges' life: examining term
limits in Venetian doges life tenure''. This paper applies rigorous statistical
tools to the computation of the influence of certain events in history of the
Republic of Venice on the (proved) fact that doges (for-live presidents of the
Venetian republic) were chosen of a certain age, which provided a \emph{natural}
limit to their terms in office.

This was proved by Smith et al.; however, in this paper we compute the exact
point when that started to happen in the history of the republic, and also
provide an alternative explanation of why this event (the so called
\emph{serrata} or closure) was so important in the history of the republic.

We consider that this falls entirely within the topic of your journal, and
expect it to be considered for publication after all reviewers are satisfied.

\end{flushleft}


% ********** fin del texto  *******************

\hspace*{8cm} Yours  \\

\centerline{\includegraphics[width=2.5cm,height=2.5cm]{../../../../Burocracia/jj_sign}}

\hspace*{8cm} Juan J. Merelo\\
\hspace*{8cm} Catedrático de Universidad (Full professor)

\end{document}

